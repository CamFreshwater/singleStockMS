% Documents setup
\documentclass[11pt]{book}

% fix for pandoc 1.14
\providecommand{\tightlist}{%
  \setlength{\itemsep}{0pt}\setlength{\parskip}{0pt}}

\usepackage{tabu} % https://tex.stackexchange.com/questions/50332/vertical-spacing-of-a-table-cell

% Location of the csas-style repository: adjust path as needed
\newcommand{\locRepo}{csas-style}

% Use the style file in the csas-style repository (res-doc.sty)
\usepackage{\locRepo/res-doc}

% Headers and footers
\lhead{Draft working paper --- Do not cite or circulate}
% \lhead{}
\rhead{}
% \rfoot{DRAFT - DO NOT CITE}

%%%% Commands for title page etc %%%%%

% Publication year
\newcommand{\rdYear}{2019}

% Publication month
\newcommand{\rdMonth}{Month}

% Report number
\newcommand{\rdNumber}{nnn}

% Region
\newcommand{\rdRegion}{Pacific Region}

% Title
\newcommand{\rdTitle}{Benefits and limitations of increasing Pacific salmon stock selectivity}

% Author names separated by commas and ', and' for the last author in the format 'M.H. Grinnell' (use \textsuperscript{n} for addresses)
\newcommand{\rdAuth}{First. M. Last\textsuperscript{1} and Alex B. Smith\textsuperscript{2}}

% Author names reversed separated by commas in the format 'Grinnell, M.H.'
\newcommand{\rdAuthRev}{Last, F.M. and Smith, A.B.}

% Author addresses (use \textsuperscript{n})
\newcommand{\rdAuthAddy}{\textsuperscript{1}Pacific Biological Station\\
Fisheries and Oceans Canada, 3190 Hammond Bay Road\\
Nanaimo, British Columbia, V9T 6N7, Canada\\
\textsuperscript{2}Far, far away\\
Another Galaxy}

\newcommand{\citationOtherLanguage}{Last, F.M. et Smith, A.B. Title Here (\emph{Latin Species Name}). DFO Secr. can. de consult. sci. du MPO. Doc. de rech 2019/nnn. iv + 13 p.}

% Name of file with abstract and resume (see \abstract and \frenchabstract for requirements)
\newcommand{\rdAbstract}{\abstract{Here is the abstract text. Lorem ipsum dolor sit amet, consectetur adipisicing elit, sed do eiusmod tempor incididunt ut labore et dolore magna aliqua. Ut enim ad minim veniam, quis nostrud exercitation ullamco laboris nisi ut aliquip ex ea commodo consequat. Duis aute irure dolor in reprehenderit in voluptate velit esse cillum dolore eu fugiat nulla pariatur. Excepteur sint occaecat cupidatat non proident, sunt in culpa qui officia deserunt mollit anim id est laborum. \break \break Start new paragraphs after a blank line and with 2 spaces indent. Lorem ipsum dolor sit amet, consectetur adipisicing elit, sed do eiusmod tempor incididunt ut labore et dolore magna aliqua. Ut enim ad minim veniam, quis nostrud exercitation ullamco laboris nisi ut aliquip ex ea commodo consequat. Duis aute irure dolor in reprehenderit in voluptate velit esse cillum dolore eu fugiat nulla pariatur. Excepteur sint occaecat cupidatat non proident, sunt in culpa qui officia deserunt mollit anim id est laborum.}}

%%%% End of title page commands %%%%%

% \pdfcompresslevel=5 % faster PNGs

\setcounter{section}{0}

\bibliographystyle{csas-style/res-doc}

\usepackage{amsmath}
\usepackage{bm}

% commands and environments needed by pandoc snippets
% extracted from the output of `pandoc -s`
%% Make R markdown code chunks work
\usepackage{array}
\usepackage{amssymb,amsmath}
\usepackage{color}
\usepackage{fancyvrb}
\DefineShortVerb[commandchars=\\\{\}]{\|}
\DefineVerbatimEnvironment{Highlighting}{Verbatim}{commandchars=\\\{\}}
% Add ',fontsize=\small' for more characters per line
\newenvironment{Shaded}{}{}
\newcommand{\KeywordTok}[1]{\textcolor[rgb]{0.00,0.44,0.13}{\textbf{{#1}}}}
\newcommand{\DataTypeTok}[1]{\textcolor[rgb]{0.56,0.13,0.00}{{#1}}}
\newcommand{\DecValTok}[1]{\textcolor[rgb]{0.25,0.63,0.44}{{#1}}}
\newcommand{\BaseNTok}[1]{\textcolor[rgb]{0.25,0.63,0.44}{{#1}}}
\newcommand{\FloatTok}[1]{\textcolor[rgb]{0.25,0.63,0.44}{{#1}}}
\newcommand{\CharTok}[1]{\textcolor[rgb]{0.25,0.44,0.63}{{#1}}}
\newcommand{\StringTok}[1]{\textcolor[rgb]{0.25,0.44,0.63}{{#1}}}
\newcommand{\CommentTok}[1]{\textcolor[rgb]{0.38,0.63,0.69}{\textit{{#1}}}}
\newcommand{\OtherTok}[1]{\textcolor[rgb]{0.00,0.44,0.13}{{#1}}}
\newcommand{\AlertTok}[1]{\textcolor[rgb]{1.00,0.00,0.00}{\textbf{{#1}}}}
\newcommand{\FunctionTok}[1]{\textcolor[rgb]{0.02,0.16,0.49}{{#1}}}
\newcommand{\RegionMarkerTok}[1]{{#1}}
\newcommand{\ErrorTok}[1]{\textcolor[rgb]{1.00,0.00,0.00}{\textbf{{#1}}}}
\newcommand{\NormalTok}[1]{{#1}}
\newcommand{\OperatorTok}[1]{\textcolor[rgb]{0.00,0.44,0.13}{\textbf{{#1}}}}
\newcommand{\BuiltInTok}[1]{\textcolor[rgb]{0.00,0.44,0.13}{\textbf{{#1}}}}
\newcommand{\ControlFlowTok}[1]{\textcolor[rgb]{0.00,0.44,0.13}{\textbf{{#1}}}}
\begin{document}

\frontmatter

\section{INTRODUCTION}\label{introduction}

Resource management is characterized by trade-offs between conservation- and harvest-based objectives. Fisheries managers, for example, are tasked with ensuring that populations remain sufficiently abundant to minimize exitinction risk and sustain intact food webs. While conservation objectives are often legally prioritized, managers are simultaneously under pressure to meet harvest objectives related to food security, cultural traditions, or economic development. To reconcile these tensions, fisheries scientists have attempted to design harvest strategies that maximize yield and minimize the risk of overexploitation. For example, management strategy evaluations can be used to develop harvest control rules in a transparent, collaborative manner, while accounting for the uncertainty ingrained in ecological systems (Punt et al. \protect\hyperlink{ref-Punt2016}{2016}).

Many fisheries are multi-stock, composed of reproductively isolated populations, which may exhibit difference in productivity or capacity, and consequently the exploitation rates they can sustain (Ricker \protect\hyperlink{ref-Ricker1958}{1958}, Hilborn \protect\hyperlink{ref-Hilborn1985}{1985}). The simplest management approach, and historically the most frequently employed, is to identify target harvest rates based on the dynamics of stock aggregates defined by management boundaries rather than population structure (Stephenson \protect\hyperlink{ref-Stephenson2002a}{2002}). If the dynamics of stocks within the fishery are relatively similar, or exploitation is sufficiently low, such a strategy may be effective. When these assumptions are not met, however, harvest rates that are sustainable at the aggregate scale may deplete less productive stocks and even result in extirpation ({\textbf{???}}, Ricker \protect\hyperlink{ref-Ricker1958}{1958}, Reiss et al. \protect\hyperlink{ref-Reiss2009}{2009}).

Differences in exploitation rates (Stephenson \protect\hyperlink{ref-Stephenson2002a}{2002}) or non-stationary productivity (Szuwalski and Hollowed \protect\hyperlink{ref-Szuwalski2016}{2016}) can result in the contribution of stocks to aggregate abundance changing through time (Hilborn et al. \protect\hyperlink{ref-Hilborn2003}{2003}). If a smaller number of populations dominate aggregate catches, or individual populations are extirpated, stock diversity will decline. Such reductions in stock diversity can have significant impacts on conservation and harvest objectives, for example by increasing spatial and temporal variability of multi-stock catches as ecological portfolio effects weaken (Hilborn et al. \protect\hyperlink{ref-Hilborn2003}{2003}, Schindler et al. \protect\hyperlink{ref-Schindler2010}{2010}, Freshwater et al. n.d.). Given uncertainty about future environmental conditions, maximizing stock diversity is a critical means of increasing long-term resilience (Anderson et al. \protect\hyperlink{ref-Anderson2015}{2015}, Link \protect\hyperlink{ref-Link2018}{2018}).

To that end, minimizing harvest rates of depleted populations in mixed-stock fisheries may be necessary to reduce overexploitation risks. One strategy is to implement highly precautionary management targets for ecological aggregates that are based on the dynamics of the least productive component stock(s) . While such a strategy will reduce the probability of overfishing across all stocks, it is likely to increase tension between conservation and socio-economic objectives as yield is lost from more productive stocks. In cases where declines in fleet size or landings have already strained fishing communities, further reductions in allowable catch may receive substantial political opposition.

One method of reconciling conservation and socio-economic objectives is to shift harvest effort towards single-stock fisheries, simultaneously reducing the risk of overexploitation of less productive stocks and minimizing harvest reductions on abundant stocks. Increasing the stock selectivity of fisheries is particularly relevant to Pacific salmon (\emph{Oncorhynchus} spp.) where stocks are highly structured due to anadromous life history strategies and natal homing, but the majority of commercial harvest occurs in mixed-stock marine fisheries. Many salmon management aggregates are composed of both abundant and depleted stocks, and in extreme cases include stocks that sustain large commercial fisheries as well as stocks of critical conservation concern. For example within the Fraser River sockeye salmon (\emph{O. nerka}) aggregate, the abundant Chilko and endangered Quesnel stocks are harvested in the same mixed-stock fishery ({\textbf{???}}).

Extensive efforts have often been made to fine-tune large, commercially important salmon fisheries to maximize their stock selectivity. Variation among stocks in migratory routes or timing are incorporated into run reconstructions, which generate stock-specific harvest rates and, ultimately, time series of spawner and recruit abundance ({\textbf{???}}). Given sufficient spatial or temporal segregation among stocks, managers can also utilize these data to develop closures to reduce fishing mortality on stocks at low abundance ({\textbf{???}}, Walters et al. \protect\hyperlink{ref-Walters2008}{2008}). When the overlap among stocks is too great for closures to be effective, stock selectivity may also be increased by moving harvest towards spawning grounds and developing terminal fisheries (Walters et al. \protect\hyperlink{ref-Walters2008}{2008}).

As knowledge of population structure improves due to greater availability of genetic stock identification techniques, it will be possible to more broadly utilize management levers that adjust the proportion of a fishery that is mixed- or single-stock. While the challenges associated with mixed-stock salmon fisheries have been widely documented ({\textbf{???}}, Hilborn \protect\hyperlink{ref-Hilborn1985}{1985}, Walters et al. \protect\hyperlink{ref-Walters2019}{2019}), the tradeoffs between conservation and socio-economic goals associated with single-stock fisheries have not been explicitly quantified. Furthermore, it is unclear how such tradeoffs may be impacted by non-stationary population dynamics and other ecological processes. En route mortality, the proportion of adult recruits dying during freshwater migrations, may have particularly severe impacts on the viability of stock-selective fisheries. Severe en route mortality events may result in 90\% mortality rates (Cooke et al. \protect\hyperlink{ref-Cooke2004}{2004}), are likely to become more severe with climate change ({\textbf{???}}), and will disprortionately impact stock-selective fisheries in terminal areas.

Here we use a stochastic closed-loop simulation model to evaluate how shifting Pacific salmon harvest from mixed- to single-stock fisheries impacts a suite of management objectives. We first develop distinct abundance-based harvest control rules for mixed- and single-stock fisheries, which could be implemented using data commonly available to Pacific salmon fisheries managers. We then evaluate the probability of achieving conservation- and catch-based metrics as allocations to each fishery change. Finally, we quantify how the trade-offs associated with shifting from mixed- to single-stock fisheries may be moderated by population productivity, as well as mortality that occurs in river during return migration. We note that the impacts of changes in allocation will vary among user groups, dramatically complicating any management process; however it was beyond the scope of the current study to fully account for socio-economic trade-offs driven by access.

\section{METHODS}\label{methods}

\subsection{2.1 Sockeye salmon biology, fisheries and data sources}\label{sockeye-salmon-biology-fisheries-and-data-sources}

The closed-loop simulation model is structured to reflect the life history of sockeye salmon and is parameterized using data from stocks within the Fraser River. Sockeye salmon is an anadromous, semelparous fish distributed throughout the northern Pacific. Populations in southern British Columbia typically rear as juveniles in freshwater lakes for one-two years, mature in the Gulf of Alaska, and return to spawn as two- to five-year olds (Burgner \protect\hyperlink{ref-Burgner1991}{1991}). Under Canada's Wild Salmon Policy (WSP), Pacific salmon status is assessed at the scale of conservation units (CUs), defined as ``a group of wild salmon sufficiently isolated from other groups that, if lost, is unlikely to recolonize naturally within an acceptable timeframe'' (DFO \protect\hyperlink{ref-DFO2005}{2005}). While CUs are not precisely equivalent to stocks, we use the latter term throughout to place our analysis in a more generic context.

The Fraser River aggregate is Canada's largest sockeye salmon run, but is increasingly vulnerable to a range of threats including anthropogenic development, overexploitation, and climate change ({\textbf{???}}, Cohen \protect\hyperlink{ref-Cohen2012}{2012}). Productivity throughout the system declined in the 1990s (Peterman and Dorner \protect\hyperlink{ref-Peterman2012}{2012}, Freshwater et al. \protect\hyperlink{ref-Freshwater2018}{2018}), resulting in frequent fishery closures and an emergency federal inquiry into the causes of the declines (Cohen \protect\hyperlink{ref-Cohen2012}{2012}). While specific stocks have shown signs of recovery in recent years, recruitment continues to be highly variable and numerous stocks remain at very low productivity levels, with seven of 24 assessed as at risk of extinction ({\textbf{???}}).

Although Fraser River sockeye salmon status is assessed at the stock level, the majority of harvest occurs in the marine environment and is managed using four run timing aggregates (management units; MUs). As a result, harvest effort can be constrained at the scale of MUs, but not individual stocks. Since MUs can include stocks ranging in status from abundant to severely depleted, increasing the stock-selectivity of harvest rates may provide an opportunity to increase the probability of rebuilding depleted stocks, while minimizing socio-economic impacts.

We focused our analysis on the summer-run MU, which contains six stocks that vary in status and cyclic behavior. The stock-recruit time series cover 31 years, begining in 1980 and ending in 2011. Note that throughout this manuscript we use the term \emph{return} abundance to refer to the sum of catch and escapement in a given return year, where escapement is defined as the number of fish that have escaped all fisheries. In comparison, we use the term \emph{recruit} abundance to refer to the sum of catch and escapement produced by spawners from a given brood year (see Grant et al. (\protect\hyperlink{ref-Grant2011}{2011}) for details).

\subsubsection{2.2 Forward simulation}\label{forward-simulation}

Closed-loop simulations provide a powerful tool to evaluate the trade-offs among objectives associated with different management frameworks and the consequences of uncertainty. Management strategy evaluation (MSE) is a formalized process for quantifying the performance of different management frameworks, which combines closed-loop simulations with extensive stakeholder and decision-maker engagement (Punt et al. \protect\hyperlink{ref-Punt2016}{2016}). Typically closed-loop simulations contain two main components. First, the operating model represents the attributes of the system that are largely beyond the immediate control of managers (e.g.~population dynamics of stocks, including declines in productivity associated with climate change). The second model component, the management procedure, represents dimensions of the fishery under more direct anthropogenic control. While examining the performance of different management procedures can be used to evaluate the impact of different harvest control rules (Holt and Peterman \protect\hyperlink{ref-Holt2008}{2008}) or the relative accuracy of assessments (Holt et al. \protect\hyperlink{ref-KHolt2011}{2011}, Holt and Folkes \protect\hyperlink{ref-Holt2015}{2015}), different operating models can be used to evaluate the impact of dynamic processes, such as changes in natural mortality and recruitment deviations (Holt \protect\hyperlink{ref-Holt2010}{2010}, Freshwater et al. n.d.). Unique combinations of operating models and management procedures, referred to here as scenarios, are used to evaluate how well a given management strategy meets \emph{a priori} objectives, given uncertainty represented by different forms of the operating model.

\subsubsection{2.2.1 Reference operating model}\label{reference-operating-model}

The dynamics of each Fraser River stock were simulated using an age-structured, Ricker spawner-recruit model (Ricker \protect\hyperlink{ref-Ricker1975}{1975})

Equation 1 \[R_{i,y} = S_{i,y} e^{\alpha_i - \beta_iS_{i,y} + w_{i,y}}\]

where \(i\) represents a stock, \(y\) is a given brood year, \(R\) the number of recruits, and \(S\) the number of spawners. The parameter \(\alpha\) represents the number of recruits produced per spawner at low abundance, while \(\beta\) is the density-dependent parameter that represents the reciprocal of the number of spawners that maximizes recruitment. The Ricker model is commonly represented in the linear form

Equation 2 \[log(\frac{R_{i,y}}{S_{i,y}}) = \alpha_i - \beta_iS_{i,y} + w_{i,y}\]

where \(w_{i,y} \sim normal(0, \sigma^2)\).

A subset of stocks within the Fraser River are cyclic, exhibiting large differences in abundance at four-year intervals. Abundant year classes, typically referred to as dominant as opposed to off-cycle, contribute disproportionately to aggregate returns (Grant et al. \protect\hyperlink{ref-Grant2011}{2011}). We used a Larkin model, an extension of the Ricker model that accounts for delayed-density dependent effects by incorporating three additional \(\beta\) parameters and spawner abundances from the preceding three years (Larkin \protect\hyperlink{ref-Larkin1971}{1971}), to simulate the dynamics of cyclic stocks (two of six stocks in the Summer Run MU; Appendix S1).

To account for observed productivity declines in Fraser River sockeye salmon (Peterman and Dorner \protect\hyperlink{ref-Peterman2012}{2012}, Freshwater et al. \protect\hyperlink{ref-Freshwater2018}{2018}), we used a recursive Bayes stock-recruitment model to estimate non-stationary \(\alpha\) parameters. To parameterize the forward simulation model we estimated median parameter sets (i.e.~median \(\alpha\) from the most recent time step and corresponding \(\beta\) and \(\sigma\)) using stock-specific spawner-recruit models (details of model structure and fitting in Appendix S1).

Year- and stock-specific recruitment deviations \(w_{i,y}\) in the forward simulation incorporated temporal autocorrelation and covariance in recruitment deviations among stocks

Equation 3a \[w_{i,y} = w_{i,y-1}\tau + r_{i,y}\] Equation 3b \[r_i \sim MVN(0, \boldsymbol{\mathrm{V}})\] Equation 3c \[\boldsymbol{\mathrm{V}} =
 \begin{bmatrix}
  \sigma'^2_1 & \rho\sigma'_1\sigma'_2 & \cdots & \rho\sigma'_1\sigma'_n \\
  \rho\sigma'_1\sigma'_2 & \sigma'^2_2 & \cdots & \rho\sigma'_2\sigma'_n \\
  \vdots  & \vdots  & \ddots & \vdots  \\
  \rho\sigma'_1\sigma'_n & \rho\sigma'_2\sigma_n & \cdots & \sigma'^2_n
 \end{bmatrix}\]
where \(w_{i,y-1}\) represents recruitment deviations one year prior, \(\tau\) represents an autoregressive lag-1 year (AR1) correlation coefficient, and \(r_{i,y}\) represents multivariate normally distributed error with a standard deviation defined by the variance-covariance matrix \(\boldsymbol{\mathrm{V}}\), dimensioned by the number of CUs \(n\). \(\rho\) represents the pairwise correlation coefficient among stocks and was set to \(0.4\) based on the mean observed correlation in recruitment deviations in recent years. We adjusted estimates of \(\sigma\) to account for \(\tau\) using the transformation, \(\sigma'^2 = \sigma^2(1-\tau^2 )\), where \(\sigma^2\) is the variance derived from a model without autocorrelation and \(\sigma'^2\) is the adjusted value (Pestal et al. \protect\hyperlink{ref-Pestal2011}{2011}). \(\tau\) was set to 0.2 based on evidence of weak autocorrelation in the residuals of stock-recruit models from preliminary analyses.

Although the majority of Fraser River sockeye salmon mature at age 4 (i.e.~one winter as eggs, one winter as fry in lakes, and two ocean winters), smaller proportions mature at ages 2, 3, and 5, with age structure varying among stocks We modeled this process by calculating the number of individuals returning to spawn \(R'_{i,t}\) in return year \(t\) in stock as a function of the total number of adult recruits \(R\) generated in previous brood years, multiplied by the proportion \(p\) of fish that return at a given age \(g\)

Equation 4 \[ R'_{i,t} = R_{i,t-2}p_{2,i,t-2} + R_{i,t-3}p_{3,i,t-3} + R_{i,t-4}p_{4,i,t-4} + R_{i,t-5}p_{5,i,t-5}. \]

\clearpage

\section*{REFERENCES}\label{references}
\phantomsection
\addcontentsline{toc}{section}{REFERENCES}
% This manually sets the header for this unnumbered chapter.
\noindent
\vspace{-2em}
\setlength{\parindent}{-0.2in}
\setlength{\leftskip}{0.2in}
\setlength{\parskip}{8pt}

\hypertarget{refs}{}
\hypertarget{ref-Anderson2015}{}
Anderson, S. C., W. M. Jonathan, M. M. Michelle, K. D. Nicholas, and B. C. Andrew. 2015. Portfolio conservation of metapopulations under climate change. Ecological Applications 25:559--572.

\hypertarget{ref-Burgner1991}{}
Burgner, R. L. 1991. Life history of Sockeye Salmon (Oncorhynchus nerka). \emph{in} C. Groot and L. Margolis, editors. Pacific salmon life histories. University of British Columbia Press, Vancouver, B.C.

\hypertarget{ref-Cohen2012}{}
Cohen, B. I. 2012. The Uncertain Future of Fraser River Sockeye - Part 1. Cohen Commission 1:692.

\hypertarget{ref-Cooke2004}{}
Cooke, S. J., S. G. Hinch, A. P. Farrell, M. F. Lapointe, S. R. M. Jones, J. S. Macdonald, D. A. Patterson, M. C. Healey, and G. van der Kraak. 2004. Abnormal migration timing and high en route mortality of sockeye salmon in the Fraser River, British Columbia. Fisheries Research 29:22--33.

\hypertarget{ref-DFO2005}{}
DFO. 2005. Canada's Policy for Conservation of Wild Pacific Salmon. Fisheries; Oceans Canada.

\hypertarget{ref-FreshwaterSub}{}
Freshwater, C., S. A. Anderson, K. R. Holt, A.-M. Huang, and C. A. Holt. (n.d.). Weakened portfolio effects constrain management effectiveness for population aggregates. Ecological Applications.

\hypertarget{ref-Freshwater2018}{}
Freshwater, C., B. J. Burke, M. D. Scheuerell, S. C. H. Grant, M. Trudel, and F. Juanes. 2018. Coherent population dynamics associated with sockeye salmon juvenile life history strategies. Canadian Journal of Fisheries and Aquatic Science 75:1346--1356.

\hypertarget{ref-Grant2011}{}
Grant, S. C. H., B. L. MacDonald, T. E. Cone, C. A. Holt, A. Cass, E. J. Porszt, J. M. B. Hume, and L. B. Pon. 2011. Evaluation of uncertainty in Fraser Sockeye (Oncorhynchus nerka) wild salmon policy status using abundance and trends in abundance metrics. Candian Science Advisory Secretariat Research Document 2011/087.

\hypertarget{ref-Hilborn1985}{}
Hilborn, R. 1985. Apparent stock recruitment relationships in mixed stock fisheries. Canadian Journal of Fisheries \& Aquatic Sciences:718--723.

\hypertarget{ref-Hilborn2003}{}
Hilborn, R., T. P. Quinn, D. E. Schindler, and D. E. Rogers. 2003. Biocomplexity and fisheries sustainability. Proceedings of the National Academy of Sciences 100:6564--6568.

\hypertarget{ref-Holt2010}{}
Holt, C. A. 2010. Will depleted populations of Pacific salmon recover under persistent reductions in survival and catastrophic mortality events? ICES Journal of Marine Science 67:2018--2026.

\hypertarget{ref-Holt2015}{}
Holt, C. A., and M. J. P. Folkes. 2015. Cautions on using percentile-based benchmarks of status for data-limited populations of Pacific salmon under persistent trends in productivity and uncertain outcomes from harvest management. Fisheries Research 171:188--200.

\hypertarget{ref-Holt2008}{}
Holt, C. A., and R. M. Peterman. 2008. Uncertainties in population dynamics and outcomes of regulations in sockeye salmon (Oncorhynchus nerka) fisheries: implications for management. Canadian Journal of Fisheries and Aquatic Sciences 65:1459--1474.

\hypertarget{ref-KHolt2011}{}
Holt, K. R., R. M. Peterman, and S. P. Cox. 2011. Trade-offs between monitoring objectives and monitoring effort when classifying regional conservation status of Pacific salmon (Oncorhynchus spp.) populations. Canadian Journal of Fisheries and Aquatic Sciences 68:880--897.

\hypertarget{ref-Larkin1971}{}
Larkin, P. A. 1971. Simulation studies of Adams River sockeye salmon (Oncorhynchus nerka). Journal Fisheries Research Board of Canada 28:1493--1502.

\hypertarget{ref-Link2018}{}
Link, J. S. 2018. System-level optimal yield: increased value, less risk, improved stability, and better fisheries. Canadian Journal of Fisheries and Aquatic Sciences 16:1--16.

\hypertarget{ref-Pestal2011}{}
Pestal, G., A.-M. Huang, and A. Cass. 2011. Updated methods for assessing harvest rules for Fraser River sockeye salmon (Oncorhynchus nerka). Can. Sci. Advis. Sec. Res. Doc. 2011/133:viii + 175.

\hypertarget{ref-Peterman2012}{}
Peterman, R. M., and B. Dorner. 2012. A widespread decrease in productivity of sockeye salmon (Oncorhynchus nerka) populations in western North America. Canadian Journal of Fisheries and Aquatic Sciences 69:1255--1260.

\hypertarget{ref-Punt2016}{}
Punt, A. E., D. S. Butterworth, C. L. de Moor, J. A. A. De Oliveira, and M. Haddon. 2016. Management strategy evaluation: best practices. Fish and Fisheries 17:303--334.

\hypertarget{ref-Reiss2009}{}
Reiss, H., G. Hoarau, M. Dickey-Collas, and W. J. Wolff. 2009. Genetic population structure of marine fish: mismatch between biological and fisheries management units. Fish and Fisheries 10:361--395.

\hypertarget{ref-Ricker1958}{}
Ricker, W. E. 1958. Maximum sustained yields from fluctuating environments and mixed stocks. Fisheries Research Board of Canada 15:991--1006.

\hypertarget{ref-Ricker1975}{}
Ricker, W. E. 1975. Computation and interpretation of biological statistics of fish populations. Fisheries Research Board of Canada Bulletin 191.

\hypertarget{ref-Schindler2010}{}
Schindler, D. E., R. Hilborn, B. Chasco, C. P. Boatright, T. P. Quinn, L. A. Rogers, and M. S. Webster. 2010. Population diversity and the portfolio effect in an exploited species. Nature 465:609--612.

\hypertarget{ref-Stephenson2002a}{}
Stephenson, R. L. 2002. Stock structure and management structure: an ongoing challenge for ICES. ICES Marine Science Symposia 215:305--314.

\hypertarget{ref-Szuwalski2016}{}
Szuwalski, C. S., and A. B. Hollowed. 2016. Climate change and non-stationary population processes in fisheries management. ICES Journal of Marine Science 73:1297--1305.

\hypertarget{ref-Walters2019}{}
Walters, C., K. English, J. Korman, and R. Hilborn. 2019. The managed decline of British Columbia's commercial salmon fishery. Marine Policy 101:25--32.

\hypertarget{ref-Walters2008}{}
Walters, C., J. Lichatowich, R. Peterman, and J. Reynolds. 2008. Report of the Skeena Independent Science Review Panel:144.

\setlength{\parindent}{0in} \setlength{\leftskip}{0in} \setlength{\parskip}{4pt}

\Appendices

\clearpage

\refstepcounter{chapter}

\starredchapter{APPENDIX~\thechapter. THE FIRST APPENDIX}\label{app:first-appendix}

Content here.

\clearpage

\refstepcounter{chapter}

\starredchapter{APPENDIX~\thechapter. THE SECOND APPENDIX, FOR FUN}\label{app:second-appendix}

More content.

\end{document}
